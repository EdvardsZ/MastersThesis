\chapter{Conclusion}

In this master's thesis, I have proposed 2 methods for combining SCVAEs with a standard VAE, which from the initial insights seemed to be promising methods for improving the standard VAE. Both of the methods were implemented and seamlessly integrated with both Gaussian VAEs and VQ-VAEs. The implementation of the methods proved to be successful. The methods were evaluated on the MNIST, CIFAR-10, and CelebA datasets. Both approaches yielded notable outcomes, showcasing significant enhancements over standard VAE performance in specific scenarios.


The first method, \methodOne{1}, demonstrated improvement in reconstruction loss for both Gaussian VAEs and VQ-VAEs. However, the overall generalization capabilities were enhanced mainly for VQ-VAEs. On the other hand, the second method, \methodTwo{2}, showed improvement in KL loss for Gaussian VAEs and VQ objective loss for VQ-VAEs, but it resulted in a deterioration of reconstruction loss for Gaussian VAEs and a slight improvement for VQ-VAEs. Thus, the research question regarding the potential improvement of generalization capabilities was partially confirmed, as overall improvement was observed only for VQ-VAEs.

%Limitations%
Despite the promising results of \methodOne{1}, its requirement for two decoders poses computational challenges, as observed during experimentation. Conversely, \methodTwo{2}, while showing potential in terms of KL loss and VQ objective loss, showed inferior performance compared to \methodOne{1} in reconstruction loss.

%Future work%

Looking ahead, future investigations could explore the application of these methods in text-to-image generation, potentially by modifying architectures such as DALL-E. Additionally,
exploring the potential of training a model with a training schedule where the model is first trained with \methodOne{1} or \methodTwo{2} and then followed by standard VAE training could potentially improve results, particularly in improving non-conditioned reconstruction.

% Overall%
Overall the work presented in this thesis shows that the proposed methods hold promise for improving the performance of VAEs in terms of generalization capabilities and reconstruction accuracy, with a notable improvement when integrated with VQ-VAEs. This suggests potential avenues for further research in the field of generative modeling, particularly in enhancing these methods and exploring their application to other generative models.


