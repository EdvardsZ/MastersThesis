\chapter{Introduction}

In recent years, Variational Autoencoders (VAEs) have emerged as a powerful tool in the field of deep learning and generative modeling. VAEs offer a principled framework for capturing complex data distributions and have found applications in diverse domains, including image generation, anomaly detection, and natural language processing. However, like any other machine learning model, VAEs have their own set of limitations and challenges.

One of the key challenges in training VAEs is achieving a balance between the reconstruction accuracy and the effectiveness of the learned latent representations. Traditional VAEs are typically trained to optimize a single objective, which often results in suboptimal performance on complex datasets with multiple sources of variation. In real-world scenarios, data often exhibits multiple underlying structures and tasks, and capturing these structures with a single VAE can be a challenging endeavor.

Multitask learning, on the other hand, is a paradigm that aims to improve model generalization and performance by simultaneously learning multiple related tasks. The idea of combining VAEs with multitask learning has recently gained attention in the machine learning community. Multitask Variational Autoencoders (MT-VAEs) extend the VAE framework to enable the joint learning of multiple latent variable spaces, each dedicated to a specific task or source of variation. This approach holds the promise of not only improving the reconstruction accuracy of VAEs but also enhancing their interpretability and generalization capabilities.

This research endeavors to investigate the effectiveness of MT-VAEs and their potential to outperform traditional VAEs in terms of data generation, representation learning, and task-specific performance. By conducting a comprehensive analysis and experimentation, we aim to shed light on the advantages and limitations of MT-VAEs and provide insights into the conditions under which they excel.

\section{Research Question}

The central research question addressed in this thesis is as follows:

\begin{quote}
\textit{Can Multitask Variational Autoencoders (MT-VAEs) enhance the performance and versatility of Variational Autoencoders (VAEs) by jointly optimizing multiple tasks, and under what conditions do MT-VAEs demonstrate superior performance in comparison to traditional VAEs?}
\end{quote}



