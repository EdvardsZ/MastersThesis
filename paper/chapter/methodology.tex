\chapter{Methods}

This chapter will outline the methods how SCVAEs and non-conditioned VAEs are integrated using multitask learning in this thesis. Initially, describe the method of merging SCVAEs with non-conditioned VAEs by expanding the SCVAEs framework with a second decoder. This method will be applied to both Gaussian VAEs and VQ-VAEs. In the following section, I will describe another method of merging SCVAEs with non-conditioned VAEs by utilizing the same encoder and decoder for both models by employing novel training strategies. This method will also be applied to both Gaussian VAEs and VQ-VAEs.

\section{Method 1}

The first method of merging SCVAEs with non-conditioned VAEs is to expand the VAEs framework with a second decoder $p_\xi(x|z,m)$, where $m$ is the conditioned information about the input data.
The first decoder is uses only the latent variable $z$ to reconstruct the input data. The second decoder is used to reconstruct the input data, given some conditioned information $m$ about the input data and the latent variable. 

In this thesis, the conditioned information $m$ will be sparse pixels from the input data. The sparse pixels will be used to condition the second decoder to reconstruct the input data. The first decoder will be used to reconstruct the input data with just the latent variable $z$. The input data $m$ will be acquired by sampling operation from the input data.

\subsection{Sampling operation}

\subsection{Application to Gaussian VAEs}

\begin{figure}[H]
    \centering 
    \centering
\scriptsize
\begin{tabular}{||c|c|c|c||}
\hline
 Method & Parameters & Reconstruction loss & KL loss \\
\hline
\textit{Baseline} & - & 0.0042 +- 1.2e-03 & 0.0018 +- 6.3e-04 \\
\hline
Multi Decoder & Exact sampling & 0.0036 +- 8.3e-04  $\downarrow$ & 0.0027 +- 8.7e-04  $\uparrow$ \\
\hline
Multi Decoder & Exact sampling, SoftAdapt & 0.0034 +- 1.9e-04  $\downarrow$ & 0.0029 +- 2.8e-03  $\uparrow$ \\
\hline
Multi Decoder & Uniform sampling & 0.0035 +- 6.5e-04  $\downarrow$ & 0.0026 +- 1.7e-03  $\uparrow$ \\
\hline
Multi Decoder & Uniform sampling, SoftAdapt & 0.0034 +- 7.3e-03  $\downarrow$ & 0.0029 +- 2.0e-02  $\uparrow$ \\
\hline
\end{tabular}

    \caption[Architecture of SCVAE1D.]%
    { 
        Architecture of Gaussian VAEs. The input $x$ is passed through the encoder with parameters $\phi$ producing the mean $\mu$ and the standard deviation $\sigma$ of the Gaussian distribution. The random variable $\epsilon$ is sampled from a standard Gaussian distribution and is used to sample $ z = \mu + \sigma \odot \epsilon$. The sampled $z$ is then passed through the decoder with parameters $\theta$ producing the output $\hat{x}$. The loss function to be minimized is the sum of the MSE reconstruction loss and the KL divergence regularization loss. 
    }\label{SCVAE1DFigure}
\end{figure}

\subsection{Application to VQ-VAEs}

\section{Method 2}


The first decoder is used to reconstruct the input data, given some conditioned information $m$ about the input data and the latent variable. The second decoder is used to reconstruct the input data with just the latent variable $z$.

and the second decoder is used to reconstruct the input data, given no information about the input data. In this thesis we will explore how we can use sparse pixels from the input data as the information that 

The SCVAE1D method and its VQ-VAE counterpart SCVQVAE1D is 





